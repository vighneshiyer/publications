\documentclass[sigplan,review,nonacm,10pt]{acmart}

\usepackage{minted}
\usepackage{makecell}
\usepackage{booktabs}
\usepackage{dblfloatfix}    % To enable figures at the bottom of page
\graphicspath{ {./figs/} }
\newcommand\aste{%
$^{(\ast)}$
}
\usepackage[subtle]{savetrees}

% https://tex.stackexchange.com/questions/655620/how-to-make-acmart-stop-complaining-about-missing-country-in-affiliation
\makeatletter
\def\@ACM@checkaffil{% Only warnings
    \if@ACM@instpresent\else
    \ClassWarningNoLine{\@classname}{No institution present for an affiliation}%
    \fi
    \if@ACM@citypresent\else
    \ClassWarningNoLine{\@classname}{No city present for an affiliation}%
    \fi
    \if@ACM@countrypresent\else
        \ClassWarningNoLine{\@classname}{No country present for an affiliation}%
    \fi
}
\makeatother

%%
%% \BibTeX command to typeset BibTeX logo in the docs
\AtBeginDocument{%
  \providecommand\BibTeX{{%
    Bib\TeX}}}

%% Rights management information.  This information is sent to you
%% when you complete the rights form.  These commands have SAMPLE
%% values in them; it is your responsibility as an author to replace
%% the commands and values with those provided to you when you
%% complete the rights form.
\setcopyright{acmcopyright}
\copyrightyear{2023}
\acmYear{2023}
%\acmDOI{XXXXXXX.XXXXXXX}

%% These commands are for a PROCEEDINGS abstract or paper.
\acmConference[PLARCH '23]{PLARCH 2023}{June 17, 2023}{Orlando, FL}
%%
%%  Uncomment \acmBooktitle if the title of the proceedings is different
%%  from ``Proceedings of ...''!
%%
%%\acmBooktitle{Woodstock '18: ACM Symposium on Neural Gaze Detection,
%%  June 03--05, 2018, Woodstock, NY}
%\acmPrice{15.00}
%\acmISBN{978-1-4503-XXXX-X/18/06}


%%
%% Submission ID.
%% Use this when submitting an article to a sponsored event. You'll
%% receive a unique submission ID from the organizers
%% of the event, and this ID should be used as the parameter to this command.
%%\acmSubmissionID{123-A56-BU3}

%%
%% For managing citations, it is recommended to use bibliography
%% files in BibTeX format.
%%
%% You can then either use BibTeX with the ACM-Reference-Format style,
%% or BibLaTeX with the acmnumeric or acmauthoryear sytles, that include
%% support for advanced citation of software artefact from the
%% biblatex-software package, also separately available on CTAN.
%%
%% Look at the sample-*-biblatex.tex files for templates showcasing
%% the biblatex styles.
%%

%%
%% The majority of ACM publications use numbered citations and
%% references.  The command \citestyle{authoryear} switches to the
%% "author year" style.
%%
%% If you are preparing content for an event
%% sponsored by ACM SIGGRAPH, you must use the "author year" style of
%% citations and references.
%% Uncommenting
%% the next command will enable that style.
%%\citestyle{acmauthoryear}
\settopmatter{printacmref=false}

%%
%% end of the preamble, start of the body of the document source.
\begin{document}

%%
%% The "title" command has an optional parameter,
%% allowing the author to define a "short title" to be used in page headers.
% \title{Elements of a Next-Gen HDL}
\title{Mixed-Abstraction HDLs and A Discussion on Other Aspects of HDL Design}

%%
%% The "author" command and its associated commands are used to define
%% the authors and their affiliations.
%% Of note is the shared affiliation of the first two authors, and the
%% "authornote" and "authornotemark" commands
%% used to denote shared contribution to the research.

\author{Vighnesh Iyer}
\email{vighnesh.iyer@berkeley.edu}
\orcid{0000-0001-6934-6577}
\affiliation{%
  \institution{UC Berkeley}
}

\author{Borivoje Nikolić}
\email{bora@eecs.berkeley.edu}
\orcid{0000-0003-2324-1715}
\affiliation{%
  \institution{UC Berkeley}
}


%%
%% By default, the full list of authors will be used in the page
%% headers. Often, this list is too long, and will overlap
%% other information printed in the page headers. This command allows
%% the author to define a more concise list
%% of authors' names for this purpose.
% \renewcommand{\shortauthors}{Trovato et al.}

% MY NOTES for this paper are in:
%   research/hdl.md
%   research/phd_planning/idea_archive/chisel.md
%   research/conf_retreats/asplos23_latchup23.md
%   PL for HW reading group notes about HDL implementation strategies (https://docs.google.com/document/d/1QxIwArsHW6GMoEqYOqNAx8P1vQu7ARf1n9dMm0lHCOE/edit)

\begin{abstract}
% In the past, HDLs have been 1) focused on one-level of abstraction (RTL, event-driven simulation, HLS), 2) not aware of cross-cutting concerns outside their domain, 3) more-or-less ad-hoc approaches to hardware design
% Let's analyze the past HDLs and try to pull new ideas to determine what the next-generation HDL ought to look like
% What is missing is a hollistic view of the design, verification, and implementation process.
% How should HDLs capture design elements from the complete flow of hardware design, verification, and fabrication.
% We will discuss problems to solve in frontend, middle-end, and backend implementations of HDLs to motivate the design of the next-generation HDL.

Many new hardware design languages (HDLs) (including freestanding DSLs and eDSLs), IRs (FIRRTL, CoreIR, CIRCT) and HDL databases (UHDM) have emerged in the past decade.
Looking forward to the next generation of HDLs, we argue that composition of hardware descriptions from multiple abstraction levels is critical to continued improvements in design productivity.
We also discuss other HDL implementation questions such as what primitives an IR should support, how design semantics should be preserved through lowering, the role of modules, and the potential for incremental elaboration.
\end{abstract}

% \received{20 February 2007}
% \received[revised]{12 March 2009}
% \received[accepted]{5 June 2009}

\maketitle

\section{Hardware Modeling Abstractions}

\begin{figure*}[!hbt]
\small
\begin{tabular}{>{\raggedright\arraybackslash}p{2cm}>{\raggedright\arraybackslash}p{3cm}>{\raggedright\arraybackslash}p{2cm}>{\raggedright\arraybackslash}p{3cm}>{\raggedright\arraybackslash}p{3cm}>{\raggedright\arraybackslash}p{2cm}}\toprule
\textbf{Abstraction} & \textbf{Primitives} & \textbf{Simulation} & \textbf{Examples} & \textbf{Utility} & \textbf{Domain} \\\midrule
%Transistor-level & PDK transistor models, wires, passives, physical geometry & Solving systems of differential equations & SPICE, Spectre & Analog circuit design, standard cell characterization & Continuous time and voltage\\\midrule
Gate-level & PDK standard cells, arbitrary time advancement & Event-driven simulation & Structural Verilog with SDF annotation & Post-syn verification, timing validation & Discrete time, 4-valued state\\\midrule
Register-transfer level & Flops, SRAMs, logic/arithmetic operators, wires & Transition system simulation & Lava\cite{lava}, Chisel\cite{chisel}, Amaranth\cite{amaranth}, PyMTL3\cite{pymtl3} & Synchronous digital circuit design and verification & Regular clock ticks, 2/4 valued signals\\\midrule
Digital discrete-event modeling & State, event-triggered blocks, delays & Event-driven simulation & Verilog, VHDL & Behavioral modeling of digital/analog IPs & Discrete time, 4-valued signals\\\midrule
Pipelines & Token streams, ordered operations, dataflow history & Ordered token processing & DFiant\cite{dfiant}, TL-Verilog\cite{tlverilog}, PipelineC\cite{pipelinec} & Dataflow circuits, auto-pipelining & Tokens in, tokens out\\\midrule
Imperative control & State, update blocks, control blocks & Interpretation & Calyx\cite{calyx}, XLS\cite{xls} & High-level synthesis & Steps in the control block\\\midrule
Transaction-level models & Ports, channels, threads, imperative code & TLM kernel & SystemC-TLM, PyMTL3 & System-level modeling, high-level synthesis & Transaction traces\\\bottomrule
\end{tabular}
\caption{Overview of abstraction levels for describing hardware designs. The boundaries of above-RTL abstractions are fuzzy.}
\label{fig:abstractions}
\end{figure*}

% Overview of multiple levels of abstraction used in real HDLs

Hardware can be modeled at various levels of abstraction spanning from low-level transistor-level constructs to high-level algorithmic constructs (see Figure \ref{fig:abstractions}).
Most modern HDLs have focused on providing a metaprogramming layer on the register-transfer level (RTL) abstraction, since this is the typical abstraction used by digital designers.
Some HDLs work at a higher-level, abstracting away explicit state instantiations or control flow logic: they can be based on guarded atomic actions (e.g. BlueSpec Verilog\cite{bluespecverilog}), decoupled dataflow pipelines (e.g. PipelineC\cite{pipelinec}, DFiant\cite{dfiant}, Shakeflow\cite{shakeflow}), highly-controllable lightweight-HLS constructs (e.g. Calyx\cite{calyx}, XLS\cite{xls}), or largely automated heavyweight-HLS from SystemC/C++ (e.g. Vivado HLS, Catapult, LegUp\cite{legup}).
Some HDLs, like Verilog, work at an arguably ``lower-level'' than RTL, opting to describe hardware as a generic discrete event simulation.

% The IR question + the benefits of rigorously defined semantics vs Verilog

While some of these HDLs directly emit Verilog or VHDL, others target an intermediate representation (IR) with well defined semantics, that is incrementally lowered by a compiler to a form resembling structural Verilog.
% only a subset of Verilog makes sense for synthesizable hardware
Relying on formally defined semantics for an IR (to check equivalence of transformations, to perform simulation\cite{hwsemantics}, and to specify undefined behavior) is an improvement\cite{cider} over relying on the semantics of Verilog, which are only well defined for specific event-based constructs that represent synthesizable hardware.

% New multi-level IRs (CIRCT)
% Have a formal semantics for each IR and how they interact with each other
% Move away from "Verilog simulation as truth", don't just be a Verilog emitter / metaprogramming layer
% Preserve the semantics of each abstraction level vs lowering things to verilog and blackboxing.

Recently, the CIRCT project\cite{circt} has developed several IRs that span RTL (the ``hw'' + ``seq'' + ``comb'' dialects) and higher-level abstractions (the ``handshake'', ``esi'', ``ssp'', and ``calyx''\cite{calyx} dialects).
While these dialects may not have formal semantics yet, there is an opportunity to precisely define them (including undefined behaviors), and move away from the idea of ``Verilog simulation as truth''.

For now, these IRs do not interact with each other, and are either simulated entirely in their own context, or are composed by, first lowering to the RTL abstraction dialect, and then connecting them.
However, there exists an opportunity to define the semantics of the \textit{composition} and \textit{interaction} of circuits expressed in different IRs.
We make the case that the next-gen HDL should support mixed-abstraction circuit definitions.

\section{Mixed-Abstraction HDLs}

% The case for mixed-abstraction/paradigm HDLs and how to leverage multi-level IRs (CIRCT)
% Mix full-blown HLS, lightweight HLS-like abstractions, RTL, and event-driven models (simulation only) in the same design
% Be able to use the abstraction most suited for your problem and mix them together
% Leverage the simulation throughput vs fidelity tradeoffs to get better simulation performance and improve productivity
% High level for algorithms and custom HLS light flows

Dataflow algorithms are more ergonomic to implement, and easier to optimize, when written in a high-level abstraction (above RTL).
Ideally, we would not write the accelerator in a different HDL, lower it to Verilog, and then simulate it with the rest of the RTL, integrated as a Verilog blackbox.
Instead, a single HDL should leverage Circt's multiple IR dialects to support mixed-abstraction simulation to yield higher simulation performance and improve integration ergonomics\cite{pymtl3, esi, hector}.
Such an approach also enables an ergonomic flow for refinement proofs from high-level models to RTL, as well as performance correlation checking between timed models and RTL.

% Low level for clocks, even lower for mixed signal, what about async resets (iffy simulation semantics leak into FIRRTL), timing simulation

An HDL should not only support mixing of abstractions above RTL, but also those below, in particular, generic event driven simulation constructs.
HDLs designed for vanilla synchronous digital logic struggle to define precise simulation semantics for common circuit features such as asynchronous resets, clock gates, and level-sensitive latches.
In simulation, clocks are a special construct, often hardcoded as a fixed frequency signal, rather than something specified in an event-driven IR that is composed with an RTL IR.
Finally, behavioral modeling of mixed-signal circuits\cite{llhd}, such as PHYs, are critical for system-level verification, and ideally would have clean semantics for interaction with vanilla RTL.
By separating different abstractions into different IRs, a simulator can understand which parts of the design need what level of simulation fidelity and flexibility.

\subsection{Mixed-Abstraction Interaction}

The interaction of high-level abstractions, such as HLS'ed blocks and software models, with RTL is relatively straightforward: transaction-level modeling and decoupled latency-insensitive interfaces provide a canonical solution\cite{umoc}.
However, a mixed-abstraction HDL should also support \textit{shared state} between the HLS'ed block and RTL.
This requires an integrated compiler approach that can understand both IRs and how they interact.

A greater challenge exists when trying to couple generic event-driven simulation constructs with RTL.
Often, one will want to execute arbitrary code in between events to process a transaction or update state.
The simulator may want to statically analyze the code (e.g. a clock generator) to make optimizations (e.g. use per-cycle simulation rather than event loop-based simulation when the clock input is a fixed frequency).
One approach may involve expressing arbitrary code as LLVM IR that is called by a CIRCT IR routine.

\section{Practical HDL Implementation Questions}

% We now quickly discuss some practical questions when implementing an HDL.

\subsection{IR Primitives}

% What primitives should exist? How can we capture the complexity of mapping to ASIC/FPGA primitives in different PDKs?

An open question is: ``how should hardware primitives be modeled in a HDL''?
Existing approaches expose common primitives (e.g. SRAMs) through HDL primitives (e.g. Chisel's SyncReadMem), which work for the common case, but are unable to cleanly express special SRAM features such as ECC or BIST or other memory elements such as CAMs.
One ``solution'' is to use a Verilog blackbox around the custom SRAM marco, but we lose understanding of the primitive's semantics.
Another ``solution'' is to model the custom macro using the available HDL primitives, and replace it with the Verilog blackbox during compilation, but we run the risk of inaccurately modeling the macro.
The mainstream approach of using Verilog on both the behavioral model and RTL side is reasonable, but is too design and process specific to be portable.
% TODO: fix this sentence below: - has nothing to do with mixed-abstraction HDLs
% Mixed-abstraction HDLs have the potential to enable description of custom primitives in a way that capture's its semantics while also validating that blackbox substitutions have the same behavior.

\subsection{Preserving Semantics Through IR Lowering}

% Xs for synthesis
% case statements, interfaces / structs
% balance simplicity of IR and pass complexity with avoiding semantics reconstruction
% lgraph claims to be very simple, but does that come at costs?

Often, there is a tension between IRs trying to minimize their primitive count to make writing passes easier, and frontends wanting to directly represent their constructs with IR primitives.
Synthesis tools have been refined over decades to recognize patterns in human written Verilog to reconstruct useful semantics (e.g. SRAM and FSM inference).
How can we preserve semantics (using Xs for boolean optimization, unique case statements, interfaces / structs) from the frontend HDL so that they are passed onto the synthesis tool as if a human had written them, while keeping passes easy to write?
If we could design a synthesis frontend from scratch, what semantics would be most valuable to preserve, and how would it differ from existing RTL representations?

\subsection{Modules: What Are They Good For?}

We often take the module construct in Verilog as integral to digital design, but it necessitates special support for features like cross-module references and regrouping for hierarchical VLSI.
Modules have three use-cases: 1) stamping out duplicate instances of a circuit, 2) providing a boundary of parameterization for circuits, 3) adding hierarchy to the design to improve UX when searching for signals.
We note the following analogies between hardware and software languages: \textit{modules : functions}, \textit{parameters : function arguments}, \textit{hierarchy : call stack}.
Few HDLs, such as Pyrope\cite{pyrope}, elide the module concept altogether, but perhaps it's time to rethink how HDLs should express instances, parameterization, and hierarchy.

\subsection{Incremental-First HDLs}

Most HDLs today, constructed as eDSLs, have no general ability to cache partial results from run-to-run, since it requires analysis of source-level dependencies.
% knowledge of source-level dependencies that lead to generated output dependencies after elaboration.
This is an area where freestanding DSLs such as Pyrope have an advantage, since source level dependency tracking for eDSLs requires analysis of arbitary code written in the host language.
There is hope that new \textit{content-addressable} languages, such as Unison\cite{unison}, will make such dependency tracking, and thus incremental elaboration viable for HDLs implemented as eDSLs.
% no idea about how a design changes from run-to-run, thus making incremental elaboration difficult.

% \section{Frontend Implementation Details}
%
% \subsection{Implementation Strategies}
%
% freestanding DSL, eDSL, custom compiler, user-level tracing (e.g. Jax)
%
% \subsubsection{Freestanding DSLs}

% - Next generation hardware design environment (Kami + LiveHD + Chisel + HLS + PyMTL + perf modeling correlation + formal + ....)
% - Modeling + perf/functional correlation automation (PyMTL3)
% - Incremental and dynamically refined simulation support
% - Type-level encoding of things (e.g. clock/reset/power domains, one-hot signals) - is this really worth it?
% - semantics preserving
% - multi-abstraction
% - SoC architecture - diplomacy
% - what type of implementation to use?
% - unified language vs unified representation (see UHDM) - approaches to polyglot systems
% - What IR to use? SSA vs graph oriented. Single abstraction vs multiple abstractions
% build system integration - fully incremental and cloud cache aware (can use cached build artifacts) - dependencies are completely specified - need to be able to generically split elaboration at function level, verilog or otherwise simulator compilation, testbench compilation, and runtime (with even runtime caching since everything should be deterministic)

% incremental first - see unison - incremental in between generations too, not only in the same generation (e.g. with memoization) - need a way to know dependencies of general purpose code!
% how to express multi-level IRs in the type system without copying between different ADTs? is there a language with support for this - perhaps with refinement types?
% SSA vs graph representation of a circuit - see LiveHD compiler for an example of strong parallelism that is difficult to do inside LLVM / MLIR
% using effect handlers to switch between different API modes - interpretation in the front-end, fully lazy compilation into a circuit, JIT'ed segments of the interpretation for good perf/compile tradeoff

% \section{Hollistic View of Hardware Design}

% \section{Features}

% \subsection{IR formats} graph vs SSA

% \subsection{Incremental-First}

% \subsubsection{Injection points}

% chisel aspects, multi-level macros

% \subsection{Visual first, Iterative Dev}

% \subsection{Refined Simulation}

% interpretation, JIT, compilation, optimization

\begin{acks}
This work was supported in part by NSF CCRI ENS Chipyard Award \#2016662, DARPA RTML program (contract FA8650-20-2-7006), and SLICE Lab industrial sponsors and affiliates Amazon, AMD, Apple, Google, Intel, and Qualcomm.
Any opinions, findings, conclusions, or recommendations in this paper are solely those of the authors and do not necessarily reflect the position or the policy of the sponsors.
\end{acks}

\bibliographystyle{ACM-Reference-Format}
\bibliography{references}

% \appendix

\end{document}
\endinput
