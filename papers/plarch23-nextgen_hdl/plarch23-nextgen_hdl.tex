\documentclass[sigplan,review,nonacm,9pt]{acmart}

\usepackage{minted}
\usepackage{makecell}
\usepackage{booktabs}
\graphicspath{ {./figs/} }
\newcommand\aste{%
$^{(\ast)}$
}
\usepackage[subtle]{savetrees}

% https://tex.stackexchange.com/questions/655620/how-to-make-acmart-stop-complaining-about-missing-country-in-affiliation
\makeatletter
\def\@ACM@checkaffil{% Only warnings
    \if@ACM@instpresent\else
    \ClassWarningNoLine{\@classname}{No institution present for an affiliation}%
    \fi
    \if@ACM@citypresent\else
    \ClassWarningNoLine{\@classname}{No city present for an affiliation}%
    \fi
    \if@ACM@countrypresent\else
        \ClassWarningNoLine{\@classname}{No country present for an affiliation}%
    \fi
}
\makeatother

%%
%% \BibTeX command to typeset BibTeX logo in the docs
\AtBeginDocument{%
  \providecommand\BibTeX{{%
    Bib\TeX}}}

%% Rights management information.  This information is sent to you
%% when you complete the rights form.  These commands have SAMPLE
%% values in them; it is your responsibility as an author to replace
%% the commands and values with those provided to you when you
%% complete the rights form.
\setcopyright{acmcopyright}
\copyrightyear{2023}
\acmYear{2023}
%\acmDOI{XXXXXXX.XXXXXXX}

%% These commands are for a PROCEEDINGS abstract or paper.
\acmConference[PLARCH '23]{PLARCH 2023}{June 17, 2023}{Orlando, FL}
%%
%%  Uncomment \acmBooktitle if the title of the proceedings is different
%%  from ``Proceedings of ...''!
%%
%%\acmBooktitle{Woodstock '18: ACM Symposium on Neural Gaze Detection,
%%  June 03--05, 2018, Woodstock, NY}
%\acmPrice{15.00}
%\acmISBN{978-1-4503-XXXX-X/18/06}


%%
%% Submission ID.
%% Use this when submitting an article to a sponsored event. You'll
%% receive a unique submission ID from the organizers
%% of the event, and this ID should be used as the parameter to this command.
%%\acmSubmissionID{123-A56-BU3}

%%
%% For managing citations, it is recommended to use bibliography
%% files in BibTeX format.
%%
%% You can then either use BibTeX with the ACM-Reference-Format style,
%% or BibLaTeX with the acmnumeric or acmauthoryear sytles, that include
%% support for advanced citation of software artefact from the
%% biblatex-software package, also separately available on CTAN.
%%
%% Look at the sample-*-biblatex.tex files for templates showcasing
%% the biblatex styles.
%%

%%
%% The majority of ACM publications use numbered citations and
%% references.  The command \citestyle{authoryear} switches to the
%% "author year" style.
%%
%% If you are preparing content for an event
%% sponsored by ACM SIGGRAPH, you must use the "author year" style of
%% citations and references.
%% Uncommenting
%% the next command will enable that style.
%%\citestyle{acmauthoryear}
\settopmatter{printacmref=false}

%%
%% end of the preamble, start of the body of the document source.
\begin{document}

%%
%% The "title" command has an optional parameter,
%% allowing the author to define a "short title" to be used in page headers.
\title{Elements of a Next-Gen HDL}
% \subtitle{}

%%
%% The "author" command and its associated commands are used to define
%% the authors and their affiliations.
%% Of note is the shared affiliation of the first two authors, and the
%% "authornote" and "authornotemark" commands
%% used to denote shared contribution to the research.

\author{Vighnesh Iyer}
\email{vighnesh.iyer@berkeley.edu}
\orcid{0000-0001-6934-6577}
\affiliation{%
  \institution{UC Berkeley}
}

\author{Borivoje Nikolić}
\email{bora@eecs.berkeley.edu}
\orcid{0000-0003-2324-1715}
\affiliation{%
  \institution{UC Berkeley}
}


%%
%% By default, the full list of authors will be used in the page
%% headers. Often, this list is too long, and will overlap
%% other information printed in the page headers. This command allows
%% the author to define a more concise list
%% of authors' names for this purpose.
% \renewcommand{\shortauthors}{Trovato et al.}


\begin{abstract}
% In the past, HDLs have been 1) focused on one-level of abstraction (RTL, event-driven simulation, HLS), 2) not aware of cross-cutting concerns outside their domain, 3) more-or-less ad-hoc approaches to hardware design
% Let's analyze the past HDLs and try to pull new ideas to determine what the next-generation HDL ought to look like
We have seen many HDLs (Lava, Chisel, PyMTL3) emerge in the past few years as well as new IRs (FIRRTL, CoreIR, Circt / MLIR) and databases (UHDM).
What is missing is a hollistic view of the design, verification, and implementation process.
We will discuss problems to solve in frontend, middle-end, and backend implementations of HDLs to motivate the design of the next-generation HDL.
\end{abstract}

% \received{20 February 2007}
% \received[revised]{12 March 2009}
% \received[accepted]{5 June 2009}

\maketitle

% - Next generation hardware design environment (Kami + LiveHD + Chisel + HLS + PyMTL + perf modeling correlation + formal + ....)
% - Modeling + perf/functional correlation automation
% - Incremental and dynamically refined simulation support
% - Mixing full-blown HLS, lightweight abstractions, RTL, and event-driven models (simulation only) in the same design
% - Type-level encoding of things (e.g. clock/reset/power domains, one-hot signals) - is this really worth it?
% - semantics preserving
% - multi-abstraction
% - SoC architecture - diplomacy
% - what type of implementation to use?
% - unified language vs unified representation (see UHDM) - approaches to polyglot systems
% - What IR to use? SSA vs graph oriented. Single abstraction vs multiple abstractions
% build system integration - fully incremental and cloud cache aware (can use cached build artifacts) - dependencies are completely specified - need to be able to generically split elaboration at function level, verilog or otherwise simulator compilation, testbench compilation, and runtime (with even runtime caching since everything should be deterministic)

% incremental first - see unison - incremental in between generations too, not only in the same generation (e.g. with memoization) - need a way to know dependencies of general purpose code!
% how to express multi-level IRs in the type system without copying between different ADTs? is there a language with support for this - perhaps with refinement types?
% SSA vs graph representation of a circuit - see LiveHD compiler for an example of strong parallelism that is difficult to do inside LLVM / MLIR
% using effect handlers to switch between different API modes - interpretation in the front-end, fully lazy compilation into a circuit, JIT'ed segments of the interpretation for good perf/compile tradeoff

\section{Hardware Modeling Abstractions}

\subsection{Abstraction Levels}
% The case for multi-paradigm / mixed-abstraction HDLs
% Leverage multi-level IRs
% Be able to use the abstraction most suited for your problem
% Have a formal semantics for each IR and how they interact with each other, move away from "Verilog simulation as truth"
% Leverage the simulation throughput vs fidelity tradeoffs to get better simulation performance and improve productivity

Hardware can be modeled at various levels of abstraction spanning from low-level transistor-level constructs to high-level algorithmic constructs (see Figure \ref{fig:abstractions}).
Most modern HDLs have focused on the RTL abstraction solely, but we make the case for an HDL with constructs at several different

We believe that 



\begin{figure*}
\small
\begin{tabular}{p{3cm}p{3cm}p{3cm}p{3cm}p{3cm}p{3cm}p{3cm}}\toprule
\textbf{Abstraction Level} & Primitives & Simulation Technique & Examples & Utility & Values & Time \\\midrule
Transistor-level & PDK transistor models, wires, passives, physical geometry & Solving systems of differential equations & SPICE, Spectre & Analog circuit design, standard cell characterization &  & \\\midrule
Gate-level & PDK standard cells & Event-driven simulation & Structural Verilog with SDF annotations & & & \\
\bottomrule
\end{tabular}
\caption{Overview of different abstraction levels for describing hardware designs and }
\label{fig:abstractions}
\end{figure*}

The vast majority of new HDLs target the register-transfer level

X's (link to previous LATTE paper from Gilbert)


What primitives should exist? How can we capture the complexity of mapping to ASIC/FPGA primitives in different PDKs?


\section{Frontend Implementation Details}

\section{Implementation Strategies}

freestanding DSL, eDSL, custom compiler, user-level tracing (e.g. Jax)

\subsection{Freestanding DSLs}

The role of the type system

\section{Hollistic View of Hardware Design}



\section{Features}
\subsection{Semantics Preserving}

\subsection{Incremental-First}

\subsubsection{Injection points}
chisel aspects, multi-level macros

\subsection{Visual first, Iterative Dev}

\subsection{Refined Simulation}

interpretation, JIT, compilation, optimization


\section{Modules: What Are They Good For?}

\begin{acks}
Research was partially funded by SLICE Lab industrial sponsors and affiliates Amazon, AMD, Apple, Google, Intel, and Qualcomm.
\end{acks}

\bibliographystyle{ACM-Reference-Format}
\bibliography{references}

% \appendix

\end{document}
\endinput
