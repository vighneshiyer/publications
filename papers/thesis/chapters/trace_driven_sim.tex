\setchapterpreamble[u]{\margintoc}
\chapter{Trace-Driven Simulation}
\labch{trace_driven_sim}

\section{Overview of Trace-Driven Simulation}

\subsection{What is a Trace?}

\subsection{Trace-Driven uArch Models}

- Old paper: Can Trace-Driven Simulators Accurately Predict Superscalar Performance? (https://pharm.ece.wisc.edu/mikko/oldpapers/ICCD.96.final+.pdf)
  - They already make the case that the answer is no and things will continue to get worse...
  - This was understood in 1996 lol

MACSim: https://github.com/gthparch/macsim
Tao (ML model for trace-driven sim): https://arxiv.org/pdf/2404.10921 (some more examples of trace-driven simulators below)

ChampSim: https://arxiv.org/pdf/2210.14324
  Example usage of ChampSim for BP/cache design: https://www.cs.sfu.ca/~ashriram/Courses/CS7ARCH/hw/hw2.html
  https://arxiv.org/html/2408.05912v1 (Correct Wrong Path - an attempt to make Champsim aware of wrong path execution effects since those are hidden in a trace, unless explicitly instrumented)

\subsection{Limitations of Trace-Driven Simulation}

- Disprove trace driven sim first
  - Start with a single-threaded baremetal application that has no IO interactions at all except DRAM
  - Then consider interaction with a DMA device
  - Then consider other IO devices e.g. NIC
  - Then consider the impact of an OS (thread multiplexing on a single hart)
  - Then consider time-dependent behavior e.g. with a timer for interrupt scheduling (for an OS)
  - Then consider the multi-threaded case
    - Some people think that multi-threaded trace-driven simulation is viable: https://ieeexplore.ieee.org/abstract/document/5762718 but this is very fishy
  - Finally consider the M:N userspace scheduling case for a IO bound application with very high concurrency. Does the 'trace' mean anything anymore? Do we capture the application-level metrics with IPC? This is what actually matters. Does the change in IPC/mispredict rate/etc actually deliver better end-to-end performance or tail latency effects?
    - See TailBench: https://people.csail.mit.edu/sanchez/papers/2016.tailbench.iiswc.pdf

- Ipc doesn't matter if it's spinning the scheduler thread, need app level metrics
- Dr traces induce 30x core slowdown and memory bw contention and cache pollution
- Fake results when fed into trace driven sim, io appears faster than reality

\section{Execution-Driven Simulation: The Only Way}

Then talk about exe sim
https://nitish2112.github.io/post/event-driven-simulation/

\section{Design for Injection (DFI) Methodology}

Then about design for injection
And then the full flow from dbt sim to ice emulation
DBT sim -> ISA-level SoC sim -> execution-driven uArch / perf sim -> RTL sim -> Palladium-ish RTL sim -> ICE
